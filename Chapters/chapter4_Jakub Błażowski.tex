\section{Jakub Błażowski}
\label{sec:Jakub Błażowski}
  \setlength{\parindent}{1cm}
  
  \textit Aby poznać bliżej z czym mamy doczynienia należy zapoznać się z historią danej rzeczy. Pierwsze wzmianki o alkoholu pochodzą z starożytnej Mezopotamii - było to piwo, którego poznano już wtedy kilka gatunków.
  
  Po raz pierwszy informacje dotyczące problemów z spożywaniem nadmiernej ilości tego napoju pozyskaliśmy z Egiptu w ok. 3000 roku p.n.e. gdzie zakazano picia w świątyniach aby uniknąć kradzierzy, gwałtów oraz nietrzeźwości. Niedługo później zakazano równierz picia alkoholu w armii egipskiej.
  
  
  
\newpage Zdjęcie pierwszego "zoobrazowanego" spożycia alkoholu (patrz Figure~\ref{fig:piwo})
  
  \begin{figure}[htbp] 
    \centering
    \includegraphics[width=1\textwidth]{Pictures/piwo.jpg} 
    \caption{Przynajmniej tak mówili fachowcy}
    \label{fig:piwo}
  \end{figure}

Pierwsze kraje korzystające z alkoholu:
\begin{enumerate}
  \item Mezopotamia
  \item Egipt
  \item Grecja
\end{enumerate}

   \newpage Table~\ref{tab:Tabela 2} Informacje
   \begin{table}[htbp]
\centering
\begin{tabular}{|l|l|}
\hline
\textbf{Okres}  & \textbf{Przełomowe odkrycia}    \\ \hline
\textbf{ok. 4000 p.n.e} & {\color[HTML]{000000} \textbf{Pierwsze udokumentowane piwo}}    \\ \hline
\textbf{XIII w.} & {\color[HTML]{000000} \textbf{Pierwszy bezwodny alkohol}}    \\ \hline
\textbf{1669 r.} & {\color[HTML]{000000} \textbf{Opracowanie produkcji spirytusu z ziemniaków}}    \\ \hline
\end{tabular}
\label{tab:Tabela 2}
\caption{Odkrycia alkoholowe.}
\end{table}
   
Obliczanie ilości spożytych % (litry):
\[A=\frac{x*0,4 + y*0,045 + z*0,15}\]
Gdzie:
\begin{itemize}
  \item ,,x'' ilość wypitej whiskey / wódki
  \item ,,y'' ilość wypitego piwa
  \item ,,z'' ilość wypitego wina
  \end{itemize}