\section{Dawid Król}
\label{sec:Dawid Król}

\textit{Wzór na pochodną z granicy}
\[F`(x)=\lim_{x\to\\h}\frac{f(x + h) - f(x)}{h}\]

\begin{flushleft}
\textbf{Shiba inu}, breed of dog that originated in Japan some \textbf{3,000} years ago for small-game and ground-bird hunting. A muscular dog, it stands 13–16 inches (33–41 cm) tall at the shoulders and weighs 20–30 pounds (9–14 kg). The shiba inu is known for its temper, perkiness, and triangularly set eyes. Its coat is short, plush, and straight, with a harsh undercoat, and may be white mixed with either red, ginger, tan, or light black. The dog’s ears are medium-sized, triangular, and held erect, and its tail is plumed and tightly curled over its back. Highly active, the shiba inu loves the outdoors and cold weather. Although the breed faced extinction during \underline{World War II}, its numbers have dramatically rebounded. The shiba inu is perhaps the most popular dog in Japan and was introduced in the United States in the 1950s. It is placed in the Non-Sporting Dog Group by the American Kennel Club.
\end{flushleft}

\begin{figure}[h!]
\centering
\includegraphics[width=8cm]{Pictures/pies.jpg}
\caption{Najpopularniejsze zdjęcie Shiby Inu}
\centering
\label{fig:dogs}
\end{figure} 

\textit{Lista nienumerowana}
\begin{itemize}
    \item[-] Pierwszy element    
    \item Drugi element
    \item Trzeci element
    \item Czwarty element
\end{itemize}

\textit{Lista numerowana}
\begin{enumerate}
  \item Pierwszy element
  \item Drugi element
  \item Trzeci element
\end{enumerate}

\begin{table}[h!]
\centering
\begin{tabular}{|l|l|c|ll}
\cline{1-3}
p & q & \multicolumn{1}{l|}{p =\textgreater q} &  &  \\ \cline{1-3}
0 & 0 & 1                                      &  &  \\ \cline{1-3}
0 & 1 & 1                                      &  &  \\ \cline{1-3}
1 & 0 & 0                                      &  &  \\ \cline{1-3}
1 & 1 & 1                                      &  &  \\ \cline{1-3}
\end{tabular}
\end{table}
