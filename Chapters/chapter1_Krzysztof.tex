\section{Krzysztof Czechowicz}
\label{sec:Krzysztof Czechowicz}
  \setlength{\parindent}{1cm}
  
  
  
  \textit{W pierwszym paragrafie} mojego rozdziału chciałbym skupić się na \textbf{wstępie do projektu} jakim jest zdalny alkomat. Zdalny alkomat będzie umożliwiał \textbf{sprawdzenie stanu trzeźwości} dowolnej osoby na podstawie informacji takich jak masa ciała badanej osoby, ilości wypitego przez nią alkoholu, czy chociażby czasu w jakim dany alkohol został spożyty. Pozwoli nam to z dużą precyzją określić chociażby czy osoba, która spożyła daną ilość alkoholu może wejść do samochodu czy raczej nie.
   
   {Warto również nakreślić w jakiej intencji pracujemy nad projektem. Mamy nadzieję, że nasza aplikacja pozwoli ograniczyć liczbę wypadków spowodowanych przez pijanych kierowców. W poniższej tabeli przedstawię statystyki z poprzednich lat. Prognozy są niepokojące ponieważ liczba wypadków z roku na rok się zwiększa.}
   
   
   Table~\ref{tab:Tabela 1} reprezentuje statystyki.
   
\begin{table}[htbp]
\centering
\begin{tabular}{|l|l|l|}
\hline
\textbf{Rok}  & \textbf{Liczba wypadków}            & \textbf{Liczba wypadków śmiertelnych} \\ \hline
\textbf{2017} & {\color[HTML]{000000} \textbf{648}} & {\color[HTML]{9A0000} \textbf{57}}    \\ \hline
\textbf{2018} & {\color[HTML]{000000} \textbf{757}} & {\color[HTML]{9A0000} \textbf{68}}    \\ \hline
\textbf{2019} & {\color[HTML]{000000} \textbf{984}} & {\color[HTML]{9A0000} \textbf{87}}    \\ \hline
\textbf{2020} & {\color[HTML]{000000} \textbf{1163}} & {\color[HTML]{9A0000} \textbf{102}}    \\ \hline
\end{tabular}
\label{tab:Tabela 1}
\caption{Statystyki wypadków z udziałem pijanych kierowców.}
\end{table}
   
\newpage Zdjęcie z ostatniego wypadku z udziałem pijanego (patrz Figure~\ref{fig:wypadek})

\begin{figure}[htbp] 
    \centering
    \includegraphics[width=0.7\textwidth]{Pictures/wypadek.png} 
    \caption{Wypadek z dnia 20 października 2021 roku.}
    \label{fig:wypadek}
\end{figure}

Poniżej przedstawie wzór który wyliczy nam ilość alkoholu we krwii badanej osoby:

\[P=\frac{x*m+h}{x*h+m^n}\]
\begin{itemize}
  \item ,,x'' oznacza procent alkohol przeliczony na alkohol dziesięcioprocentowy w litrach.
  \item ,,m'' oznacza masę osoby badanej.
  \item ,,h'' oznacza wysokość osoby badanej.
\end{itemize}
Z kolei liczba $ n=g^x $ może oznaczać:
\begin{enumerate}
  \item Stężenie alkoholu podniesione do potęgi spożytej ilości alkoholu w ml.
  \item Ilość czystego alkoholu
  \item Czas w jakim alkohol podany w powyżej wymienionym wzorze został spożyty.
\end{enumerate}