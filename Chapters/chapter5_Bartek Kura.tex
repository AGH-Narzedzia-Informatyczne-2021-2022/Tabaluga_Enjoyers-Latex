\section{Bartłomiej Kura}
\label{sec:Bartłomiej Kura}
\setlength{\parindent}{1cm}



    \textbf     {Pustak ścienny}– wyrób \textit{ceramiczny} lub \textit{betonowy} (z betonu zwykłego lub na kruszywie lekkim, np. keramzycie) przeznaczony do wykonywania ścian zewnętrznych i wewnętrznych. Proces produkcji pustaków ceramicznych jest analogiczny do produkcji cegły ceramicznej, pustaki betonowe wytwarza się przez wypełnienie form masą betonu, zagęszczenie, rozformowanie. Pustaki poprzez proces wiązania i twardnienia uzyskują wymaganą wytrzymałość mechaniczną.\par
    Pustaki charakteryzują się zazwyczaj większymi wymiarami i otworami o różnym układzie. Otwory często usytuowane są mijankowo, co minimalizuje powstawanie mostków termicznych. Większe gabaryty  pustaków przyśpieszają wykonywanie robót murarskich oraz zmniejszają liczbę spoin, które także mają mniejszą izolacyjność niż pustak. Izolacyjność termiczna ścian jest najistotniejsza przy wykonywaniu ścian zewnętrznych. Do wyrobów ceramicznych należą także specjalne pustaki do murowania przewodów dymowych. Pustaki te, z zewnątrz sześcienne, mają wewnątrz przelotowy, okrągły otwór. Pustaki, podobnie jak i inne materiały używane do wykonywania elementów konstrukcyjnych budynku produkowane są o różnej, określonej wytrzymałości mechanicznej. Głównymi producentami pustaków są Niemcy i Polska(\ref{tab:produkcja}).


   
   
   
Objętość pustaka liczy się ze wzoru:
\[ V = a*b*c \]

Gdzie:
\begin{itemize}
  \item \textit{a} to jeden z wymiarów podstawy
  \item \textit{b} to drugi z wymiarów podstawy
  \item \textit{c} to wysokość pustaka
\end{itemize}

Rodzaje pustaków:
\begin{enumerate}
  \item Pustak stropowy
  \item Pustak ścienny
  \item Pustak styropianowy
  \item Pustak szklany
\end{enumerate}



   
\begin{figure}[htbp]
    \centering
    \includegraphics[width=0.5\textwidth]{Pictures/pustak.jpg}
    \caption{Pustak ścienny}
\end{figure}

\begin{table}[htbp]
\centering
\begin{tabular}{|c|c|c|}
\hline
\textbf{Rok} & \textbf{Polska} & \textbf{Niemcy} \\ \hline
2010         & 115 tys.        & 321 tys.        \\ \hline
2015         & 243 tys.        & 450 tys.        \\ \hline
2020         & 331 tys.        & 519 tys.        \\ \hline
\end{tabular}
\label{tab:produkcja}
\caption {Ilość kupowanych pustaków w Polsce i Niemczech rocznie:}
\end{table}