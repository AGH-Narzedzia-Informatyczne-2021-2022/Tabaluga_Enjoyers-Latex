\section{Dawid Kmak}
\label{sec:Dawid Kmak}
  \setlength{\parindent}{1cm}
  
  
  
  \textit{ Funkcje Trygonometryczne - funkcje matematyczne, wyrażające między innymi stosunki między długościami boków trójkąta prostokątnego względem miar jego kątów wewnętrznych, będące przedmiotem badań trygonometrii.

Funkcje trygonometryczne, choć wywodzą się z pojęć geometrycznych, są rozpatrywane także w oderwaniu od geometrii. W analizie matematycznej są one definiowane m.in. za pomocą szeregów potęgowych[1] lub jako rozwiązania pewnych równań różniczkowych.

Do funkcji trygonometrycznych współcześnie zalicza się: sinus, cosinus (inna pisownia: kosinus), tangens, cotangens (kotangens), secans (sekans), cosecans (kosekans), z czego dwóch ostatnich obecnie rzadko się używa.

Funkcje trygonometryczne znajdują zastosowanie w wielu działach matematyki, innych naukach ścisłych i technice; działem matematyki badającym te funkcje jest trygonometria, lub ściślej: goniometria.}
   
   
   
   \vspace{20}
   
   Tabela~\ref{tab:Tabela 4} pokazuję wartości dla wybranych kątów.
   \begin{table}[htbp]
\begin{tabular}{c|c|c|c|c|}
\cline{2-5}
                          & sin & cos & tg & ctg  \\ \hline
\multicolumn{1}{|c|}{0°}  & 0   & 1   & 0  & brak \\ \hline
\multicolumn{1}{|c|}{30°} & $\frac{1}{2}$   & $\frac{\sqrt{3}}{2}$   & $\frac{\sqrt{3}}{3}$  & $\sqrt{3}$    \\ \hline
\multicolumn{1}{|c|}{45°} & $\frac{\sqrt{2}}{2}$   & $\frac{\sqrt{2}}{2}$   & 1  & 1    \\ \hline
\multicolumn{1}{|c|}{60°} & $\frac{\sqrt{3}}{2}$   & $\frac{1}{2}$   & $\sqrt{3}$  & $\frac{\sqrt{3}}{3}$    \\ \hline
\multicolumn{1}{|c|}{90°} & 1   & 0   & brak  & 0    \\ \hline
\end{tabular}
\label{tab:Tabela 4}
\end{table}


   
\newpage Definicja na okręgu jednostkowym (patrz Figure~\ref{fig:funkcja})

\begin{figure}[htbp] 
    \centering
    \includegraphics[width=0.7\textwidth]{Pictures/funkcja.png} 
    \caption{Funkcje trygonometryczne}
    \label{fig:funkcja}
\end{figure}

Definicja z elementów trójkąta prostokątnego:

\begin{itemize}
  \item ,,sin'' - oznaczany w Polsce jako sin  – stosunek długości przyprostokątnej a leżącej naprzeciw kąta alpha  i długości przeciwprostokątnej  c;
  \item ,,cos'' - oznaczany w Polsce jako cos  – stosunek długości przyprostokątnej b przyległej do kąta alpha  i przeciwprostokątnej c
  \item ,,tg'' - oznaczany w Polsce jako tg – stosunek długości przyprostokątnej a leżącej naprzeciw kąta alpha  i długości przyprostokątnej b przyległej do tego kąta
  \item ,,ctg'' - oznaczany w Polsce jako ctg – stosunek długości przyprostokątnej b przyległej do kąta alpha  i długości przyprostokątnej a leżącej naprzeciw tego kąta 
\end{itemize}
Wybrane wzory:
\begin{enumerate}
    \item $sin(x)=\frac{a}{c}$
    \item $cos(x)=\frac{b}{c}$
    \item $tg(x)=\frac{b}{a}$
    \item $ctg(x)=\frac{a}{b}$
    \item $sin(x)+sin(y)=2sin\frac{x+y}{2}cos\frac{x-y}{2}$
\end{enumerate}


